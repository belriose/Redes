\section{Desarrollo}

Una vez implementado el protocolo procedimos a probarlo enviando archivos, variando el tamaño de la ventana de emisión y el tamaño del archivo.	 \\
\indent	Las pruebas fueron realizadas en una red local con equipos conectados por wifi, realizando multiples experimentaciones para el mismo SEND\_WINDOW y tamaño de archivo.	\\

\subsection{Ventana de emisión}

La ventana de emisión es de suma importancia en el testeo y análisis del protocolo debido a la siguiente limitación: $throughput \leq SEND\_WINDOW / RTT$.	\\
Si bien una mayor ventana de emisión nos puede permitir utilizar la capacidad completa del canal y aumentar el throughput, si es demasiado grande, esto puede conllevar desventajas: 
\begin{itemize}
\item	Al vernos obligados a mantener los segmentos no reconocidos, en caso de ser necesarios para una posible retransmición, necesitamos un RETRANSMISSION\_BUFFER más grande.
\item	El receptor se puede ver obligado a descartar mensajes si su buffer se encuentra lleno y el emisor sigue enviando mensajes.
\item	La red o buffers intermedios pueden ser congestionados de forma que se pierdan paquetes.
\end{itemize}

\subsection{Tamaño del archivo}

Para probar la efectividad del protocolo variamos el tamaño del archivo de forma de mantener SEND\_WINDOW 
paquetes en vuelo y ver la efectividad del protocolo cuando es utilizado en toda su capacidad por períodos largos.	\\
\indent	Variando el tamaño del archivo esperamos, de estabilizarse o presentarse algún patrón en la taza de reenvios, connection timeouts o throughput, poder detectarlo.

\subsection{Retransmiciones y connection timeouts}

Un factor determinante en el desempeño del protocolo son los timeouts y la cantidad de retransmiciones realizadas debido a el impacto que tiene sobre el throughput y a que de alcanzarse los MAX RETRANSMISSION ATTEMPTS para algún paquete, la conexión es cerrada.	\\
\indent	Por este motivo se registro la cantidad de retransmiciones realizada en cada experimento y la cantidad de veces que se alcanzaron los MAX RETRANSMISSION ATTEMPTS.

\subsection{Throughput}

Con los valores obtenidos de los experimentos realizados medimos el throughput percibido y observamos la relación entre este y las retransmiciones con el fin de determinar de que forman afectan las variables testeadas en el resultado del protocolo.
