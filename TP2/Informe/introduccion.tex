\section{Introducción}

En este taller nos proponemos experimentar con herramientas y técnicas frecuentes a nivel de red. Más particularmente nos centraremos en dos muy conocidas y utilizadas: \texttt{ping} y \texttt{traceroute}. El objetivo es entender los protocolos involucrados. Para ello, desarrollaremos nuestras propias implementaciones de las herramientas de manera de afianzar los conocimientos. Todo lo anterior se realizará en un marco analítico que nos permitirá razonar sobre lo hecho y comprender mejor qué pasa detrás de bambalinas.

\subsection{ICMP}

El Protocolo de Mensajes de Control de Internet o ICMP es el sub protocolo de control y notificación de errores del Protocolo de Internet (IP). Como tal, se usa para enviar mensajes de error, indicando por ejemplo que un servicio determinado no está disponible o que un router o host no puede ser localizado.

ICMP difiere del propósito de TCP y UDP ya que generalmente no se utiliza directamente por las aplicaciones de usuario en la red. La única excepción es la herramienta \texttt{ping} y \texttt{traceroute}, que envían mensajes de petición Echo ICMP (y recibe mensajes de respuesta Echo) para determinar si un host está disponible, el tiempo que le toma a los paquetes en ir y regresar a ese host y cantidad de hosts por los que pasa.

\vspace*{5 mm}

Los mensajes ICMP son comúnmente generados en respuesta a errores en los datagramas de IP o para diagnóstico y ruteo. IP encapsula el mensaje ICMP apropiado con una nueva cabecera IP (para obtener los mensajes de respuesta desde el host original que envía), y transmite el datagrama resultante de manera habitual.

Por ejemplo, cada router que reenvía un datagrama IP tiene que disminuir el campo de tiempo de vida (TTL) de la cabecera IP en una unidad; si el TTL llega a 0, un mensaje ICMP ''Tiempo de Vida se ha excedido en transmitirse'' es enviado a la fuente del datagrama. Cada mensaje ICMP es encapsulado directamente en un solo datagrama IP, y por tanto no garantiza la entrega del ICMP. Aunque los mensajes ICMP son contenidos dentro de datagramas estándar IP, los mensajes ICMP se procesan como un caso especial del procesamiento normal de IP, algo así como el procesamiento de un sub-protocolo de IP. En muchos casos es necesario inspeccionar el contenido del mensaje ICMP y entregar el mensaje apropiado de error a la aplicación que generó el paquete IP original, aquel que solicitó el envío del mensaje ICMP.

\vspace*{5 mm}

La utilidad del protocolo ICMP es controlar si un paquete no puede alcanzar su destino, si su vida ha expirado, etc. Es decir, se usa para manejar mensajes de error y de control necesarios para los sistemas de la red, informando con ellos a la fuente original para que evite o corrija el problema detectado.

Muchas de las utilidades de red comunes están basadas en los mensajes ICMP. El comando \texttt{traceroute} está implementado transmitiendo datagramas UDP con campos especiales TTL IP en la cabecera, y buscando los mensajes de ''Tiempo de Vida en tránsito'' y ''Destino inalcanzable'' generados como respuesta. La herramienta \texttt{ping} está implementada utilizando los mensajes ''Echo request'' y ''Echo reply'' de ICMP.

\subsection{Ping}

Como programa, \texttt{ping} es una utilidad de diagnóstico en redes de computadoras que comprueba el estado de la comunicación del host local con uno o varios equipos remotos de una red TCP/IP por medio del envío de paquetes ICMP de solicitud y de respuesta. Mediante esta utilidad puede diagnosticarse el estado, velocidad y calidad de una red determinada.

\vspace*{5 mm}

Ejecutando \texttt{ping} de solicitud, el Host local envía un mensaje ICMP, incrustado en un paquete IP. El mensaje ICMP de solicitud incluye, además del tipo de mensaje y el código del mismo, un número identificador y una secuencia de números, de 32 bits, que deberán coincidir con el mensaje ICMP de respuesta; además de un espacio opcional para datos. Muchas veces se utiliza para medir la latencia o tiempo que tardan en comunicarse dos puntos remotos.

\subsection{Traceroute}

\texttt{traceroute} es una consola de diagnóstico que permite seguir la pista de los paquetes que vienen desde un host (punto de red). Se obtiene además una estadística del RTT o latencia de red de esos paquetes, lo que viene a ser una estimación de la distancia a la que están los extremos de la comunicación. 

\vspace*{5 mm}

Entre los datos que se obtienen están: el número de salto, el nombre y la dirección IP del nodo por el que pasa y el tiempo de respuesta para los paquetes enviados (un asterisco indica que no se obtuvo respuesta).

\texttt{traceroute} utiliza el campo Time To Live (TTL) de la cabecera IP. Este campo sirve para que un paquete no permanezca en la red de forma indefinida (por ejemplo, debido a la existencia en la red de un bucle cerrado en la ruta). El campo TTL es un número entero que es decrementado por cada nodo por el que pasa el paquete. De esta forma, cuando el campo TTL llega al valor 0 ya no se reenviará más, sino que el nodo que lo esté manejando en ese momento lo descartará. Lo que hace \texttt{traceroute} es mandar paquetes a la red de forma que el primer paquete lleve un valor TTL=1, el segundo un TTL=2, etc. De esta forma, el primer paquete será eliminado por el primer nodo al que llegue (ya que éste nodo decrementará el valor TTL, llegando a cero). Cuando un nodo elimina un paquete, envía al emisor un mensaje de control especial indicando una incidencia. \texttt{traceroute} usa esta respuesta para averiguar la dirección IP del nodo que desechó el paquete, que será el primer nodo de la red. La segunda vez que se manda un paquete, el TTL vale 2, por lo que pasará el primer nodo y llegará al segundo, donde será descartado, devolviendo de nuevo un mensaje de control. Esto se hace de forma sucesiva hasta que el paquete llega a su destino.
